\chapter{Introducción}

La programación competitiva combina dos asuntos:
(1) el diseño de algoritmos y (2) la implementación de algoritmos.

El \key{diseño de algoritmos} consiste de la resolución de problemas
y pensamiento matemático.
Se necesitan habilidades para analizar y resolver problemas creativamente.
Un algoritmo para resolver un problema
tiene que ser correcto y eficiente,
y a menudo la esencia del problema requiere
la invención de un algoritmo eficiente.

El conocimiento teórico de los algoritmos
es importante para los programadores competitivos.
Normalmente, una solución a un problema es
una combinación de técnicas conocidas y de
nuevas ideas. 
Las técnicas que aparecen en la programación competitiva
también forman la base para la investigación científica
de algoritmos.

La \key{implementación de algoritmos} requiere de buenas
habilidades en programación.
En la programación competitiva, las soluciones
son evaluadas utilizando un conjunto de casos de prueba
sobre un algoritmo implementado.
Por lo tanto, no es suficiente con que la idea del
algoritmo sea correcta, la implementación también
tiene que serlo.

En los concursos se utiliza un estilo de codificación sencillo y conciso.
Los programas deben escribirse rápidamente,
porque no hay mucho tiempo disponible.
A diferencia de la ingeniería de software tradicional,
los programas son cortos (cuando mucho 
unos cientos de líneas de código), y no necesitan
mantenimiento después del concurso.

\section{Lenguajes de programación}

\index{Lenguaje de programación}

Al momento, los lenguajes de programación más populares
usados en los concursos son C++, Python y Java.
Por ejemplo, en el Google Code Jam 2017,
entre los 3,000 mejores participantes,
79 \% usaron C++,
16 \% usaron Python y
8 \% usaron Java \cite{goo17}.
Algunos participantes también utilizaron varios lenguajes de programación.

Mucha gente piensa que C++ es la mejor opción
para un programador competitivo,
y C++ casi siempre está disponible en todas
las plataformas de concursos.
Los beneficios de usar C++ son que
es un lenguaje muy eficiente y
su biblioteca estándar contiene una
gran colección
de estructuras de datos y algoritmos.

Por otro lado, es bueno
dominar varios lenguajes y comprender
sus puntos fuertes.
Por ejemplo, si se necesitan números enteros grandes
en el problema,
Python puede ser una buena opción, porque
incorpora operaciones específicas para
realizar cálculos con números enteros grandes.
Aún así, la mayoría de los problemas en los concursos de programación
están configurados para que
el uso de un lenguaje de programación específico
no sea una ventaja injusta.

Todos los programas de ejemplo de este libro están escritos en C++,
y a menudo se utilizan las estructuras de datos y algoritmos 
de su biblioteca estándar.
Los programas siguen el estándar C++11,
que se puede utilizar en la mayoría de los concursos hoy en día.
Si el lector aún no sabe programar en C++,
ahora es un buen momento para empezar a aprender.

\subsubsection{Plantilla de código de C++}

La siguiente es una típica plantilla de C++
en programación competitiva:

\begin{lstlisting}
#include <bits/stdc++.h>

using namespace std;

int main() {
    // aqui viene la solucion
}
\end{lstlisting}

La línea \texttt{\#include} al inicio
del código es una característica del compilador \texttt{g++}
que nos permite incluir toda la biblioteca estándar.
Por lo tanto, no es necesario incluir bibliotecas 
por separado tales como \texttt{iostream},
\texttt{vector} y \texttt{algorithm},
sino que están disponibles automáticamente.

La línea \texttt{using} declara
que las clases y funciones
de la biblioteca estándar se pueden utilizar directamente
en el código.
Sin la línea \texttt{using} tendríamos que
escribir, por ejemplo, \texttt{std::cout},
pero ahora basta con escribir \texttt{cout}.

El código se puede compilar usando el siguiente comando:

\begin{lstlisting}
g++ -std=c++11 -O2 -Wall test.cpp -o test
\end{lstlisting}

Este comando produce el archivo binario \texttt{test}
a partir del código fuente \texttt{test.cpp}.
El compilador sigue el estándar C++11
(\texttt{-std=c++11}),
optimiza el código (\texttt{-O2})
y muestra advertencias sobre posibles errores  (\texttt{-Wall}).

\section{Entrada y salida}

\index{entrada y salida}

En la mayoría de los concursos, los flujos estándar se utilizan
para leer la entrada y escribir la salida.
En C++, los flujos estándar son
\texttt{cin} para entrada y \texttt{cout} para salida.
Además, se pueden utilizar las funciones
\texttt{scanf} y \texttt{printf} del lenguaje C.

La entrada del programa generalmente consiste en
números y cadenas que están separados por
espacios y nuevas líneas.
Se pueden leer desde el flujo de entrada \texttt{cin}
de la siguiente manera:

\begin{lstlisting}
int a, b;
string x;
cin >> a >> b >> x;
\end{lstlisting}

Este tipo de código siempre funciona,
asumiendo que hay al menos un espacio
o nueva línea entre cada elemento de la entrada.
Por ejemplo, el código anterior puede leer
las siguientes entradas de la misma manera:
\begin{lstlisting}
123 456 monkey
\end{lstlisting}
\begin{lstlisting}
123    456
monkey
\end{lstlisting}
El flujo \texttt{cout} se utiliza para la salida
de la siguiente manera:
\begin{lstlisting}
int a = 123, b = 456;
string x = "monkey";
cout << a << " " << b << " " << x << "\n";
\end{lstlisting}

A veces, la entrada y la salida
son un cuello de botella en el programa.
Las siguientes líneas al inicio del programa
hacen que la entrada y la salida sean más eficientes:

\begin{lstlisting}
ios::sync_with_stdio(0);
cin.tie(0);
\end{lstlisting}

Se debe tener en cuenta que la nueva línea \texttt{"\textbackslash n"}
funciona más rápido que \texttt{endl},
porque \texttt{endl} siempre causa una
operación de vaciado del búfer.

Las funciones de C \texttt{scanf}
y \texttt{printf} son una alternativa
a los flujos estándar de C++.
Suelen ser un poco más rápidas
pero también son más difíciles de usar.
El siguiente código lee dos números enteros de la entrada:
\begin{lstlisting}
int a, b;
scanf("%d %d", &a, &b);
\end{lstlisting}
El siguiente código imprime dos enteros:
\begin{lstlisting}
int a = 123, b = 456;
printf("%d %d\n", a, b);
\end{lstlisting}

A veces, el programa debe leer una línea completa
de la entrada, posiblemente conteniendo espacios.
Esto se puede lograr utilizando
la función \texttt{getline}:

\begin{lstlisting}
string s;
getline(cin, s);
\end{lstlisting}

Si se desconoce la cantidad de datos, el siguiente
ciclo es muy útil:
\begin{lstlisting}
while (cin >> x) {
    // codigo
}
\end{lstlisting}
Este ciclo lee elementos de la entrada
uno tras otro, hasta que no haya
más datos disponibles en la entrada.

Algunas plataformas de concursos utilizan archivos para
entrada y salida.
Una solución fácil para esto es escribir
el código como de costumbre usando flujos estándar,
pero agregando las siguientes líneas al principio del código:
\begin{lstlisting}
freopen("input.txt", "r", stdin);
freopen("output.txt", "w", stdout);
\end{lstlisting}
Después de esto, el programa lee la entrada del archivo
''input.txt'' y escribe la salida en el archivo
''output.txt''.

\section{Trabajando con números}

\index{entero}

\subsubsection{Enteros}

El tipo de entero más utilizado en programación competitiva
es \texttt{int}, el cual consta de 32 bits que proporcionan
un rango de valores de $-2^{31} \ldots 2^{31}-1$
o cerca de $-2 \cdot 10^9 \ldots 2 \cdot 10^9$.
Si el tipo \texttt{int} no es suficiente,
se puede utilizar el tipo \texttt{long long} de 64 bits.
Éste tiene un rango de valores de $-2^{63} \ldots 2^{63}-1$
o cerca de $-9 \cdot 10^{18} \ldots 9 \cdot 10^{18}$.

El siguiente código define una variable
\texttt{long long}:
\begin{lstlisting}
long long x = 123456789123456789LL;
\end{lstlisting}
El sufijo \texttt{LL} significa que el
tipo de dato del número es \texttt{long long}.

Un error común cuando se usa el tipo \texttt{long long}
es que el tipo \texttt{int} todavía se utilice 
en algún otro lugar.
Por ejemplo, el siguiente código continue
un error muy sutil:

\begin{lstlisting}
int a = 123456789;
long long b = a*a;
cout << b << "\n"; // -1757895751
\end{lstlisting}

Aunque la variable \texttt{b} es de tipo \texttt{long long},
ambos números en la expresión \texttt{a*a}
son de tipo \texttt{int} y el resultado
también es de tipo \texttt{int}.
Debido a esto, la variable \texttt{b}
contendrá un resultado equivocado.
El problema se puede resolver cambiando el tipo
de la variable \texttt{a} por \texttt{long long} o
cambiando la expresión por \texttt{(long long)a*a}.

Por lo general, los problemas del concurso se establecen de manera que
el tipo \texttt{long long} sea suficiente.
Aun así, es bueno saber que
el compilador \texttt{g++} también provee
un tipo \texttt{\_\_int128\_t} de 128 bits
con un rango de valores de
$-2^{127} \ldots 2^{127}-1$ o cerca de $-10^{38} \ldots 10^{38}$.
Sin embargo, este tipo de dato no está disponible en todas las plataformas de concursos.

\subsubsection{Aritmética modular}

\index{residuo}
\index{aritmética modular}

Denotamos a $x \bmod m$ como el residuo
cuando $x$ es dividido por $m$.
Por ejemplo, $17 \bmod 5 = 2$,
porque $17 = 3 \cdot 5 + 2$.

A veces, la respuesta a un problema es un
un número muy grande pero es suficiente
imprimirlo ''módulo $m$'', p.ej.,
el residuo cuando la respuesta es dividida por $m$
(por ejemplo, ''módulo $10^9+7$'').
La idea es que incluso si la respuesta
es muy grande,
es suficiente utilizar los tipos
\texttt{int} y \texttt{long long}.

Una propiedad importante del residuo es que
en las operaciones de suma, resta y multiplicación,
el residuo se puede realizar antes de la operación:

\[
\begin{array}{rcr}
(a+b) \bmod m & = & (a \bmod m + b \bmod m) \bmod m \\
(a-b) \bmod m & = & (a \bmod m - b \bmod m) \bmod m \\
(a \cdot b) \bmod m & = & (a \bmod m \cdot b \bmod m) \bmod m
\end{array}
\]

Por lo tanto, podemos aplicar el residuo después de cada operación
y los números nunca serán demasiado grandes.

Por ejemplo, el siguiente código calcula $n!$,
el factorial de $n$, módulo $m$:
\begin{lstlisting}
long long x = 1;
for (int i = 2; i <= n; i++) {
    x = (x*i)%m;
}
cout << x%m << "\n";
\end{lstlisting}

Por lo general, queremos que el residuo siempre
esté entre $0\ldots m-1$.
Sin embargo, en C++ y otros lenguajes,
el residuo de un número negativo
es cero o negativo.
Una forma sencilla de asegurarse
que no hay residuos negativos es primero calcular
el residuo como siempre y sumarle $m$
si el resultado es negativo:
\begin{lstlisting}
x = x%m;
if (x < 0) x += m;
\end{lstlisting}
Sin embargo, esto solo es necesario cuando hay
restas en el código y
el residuo puede volverse negativo.

\subsubsection{Números de punto flotante}

\index{número de punto flotante}

Los tipos de dato de punto flotante más comunes en
programación competitiva son
el \texttt{double} de 64 bits
y, como una extensión en el compilador \texttt{g++},
el \texttt{long double} de 80 bits.
En la mayoría de los casos, \texttt{double} es suficiente,
pero \texttt{long double} es más preciso.

La precisión requerida de la respuesta 
generalmente se da en el enunciado del problema.
Una forma sencilla de generar la respuesta es utilizar
la función \texttt{printf}
y dar el número de lugares decimales
en la cadena de formato.
Por ejemplo, el siguiente código imprime
el valor de $x$ con 9 decimales:

\begin{lstlisting}
printf("%.9f\n", x);
\end{lstlisting}

Una dificultad al usar números de punto flotante
es que algunos números no se pueden representar
exactamente como tal,
y habrá errores de redondeo.
Por ejemplo, el resultado del siguiente código
es sorprendente:

\begin{lstlisting}
double x = 0.3*3+0.1;
printf("%.20f\n", x); // 0.99999999999999988898
\end{lstlisting}

Debido a un error de redondeo,
el valor de \texttt{x} es un poco menor que 1,
mientras que el valor correcto sería 1.

Es arriesgado comparar números de punto flotante
con el operador \texttt{==},
porque es posible que aunque los valores son iguales 
al final de cuentas no lo sean debido a errores de precisión.
Una mejor manera de comparar números de punto flotante
es asumir que dos números son iguales
si la diferencia entre ellos es menor que $\varepsilon$,
donde $\varepsilon$ es un número pequeño.

En la práctcia, los números se pueden comparar
de la siguiente manera ($\varepsilon=10^{-9}$):

\begin{lstlisting}
if (abs(a-b) < 1e-9) {
    // a y b son iguales
}
\end{lstlisting}

Se debe tener en cuenta que aunque los números de punto flotante son inexactos,
los números enteros todavía pueden ser
representados con precisión hasta un cierto límite.
Por ejemplo, usando \texttt{double},
es posible representar con precisión todos
los enteros cuyo valor absoluto sea como máximo $2^{53}$.

\section{Abreviando el código}

El código abreviado es ideal en la programación competitiva,
porque los programas deben ser escritos
tan rápido como sea posible.
Debido a esto, los programadores competitivos a menudo definen
nombres más cortos para tipos de datos y otras partes del código.

\subsubsection{Nombres de tipo}
\index{tuppdef@\texttt{typedef}}
Usando el comando \texttt{typedef}
es posible dar un nombre más corto
a un tipo de dato.
Por ejemplo, el nombre \texttt{long long} es largo,
así que podemos definir un nombre más corto \texttt{ll}:
\begin{lstlisting}
typedef long long ll;
\end{lstlisting}
Después de esto, el código
\begin{lstlisting}
long long a = 123456789;
long long b = 987654321;
cout << a*b << "\n";
\end{lstlisting}
se puede abreviar de la siguiente manera:
\begin{lstlisting}
ll a = 123456789;
ll b = 987654321;
cout << a*b << "\n";
\end{lstlisting}

El comando \texttt{typedef}
también se puede utilizar con tipos más complejos.
Por ejemplo, el siguiente código establece
el nombre \texttt{vi} para un vector de enteros
y el nombre \texttt{pi} para un par
que contiene dos enteros.
\begin{lstlisting}
typedef vector<int> vi;
typedef pair<int,int> pi;
\end{lstlisting}

\subsubsection{Macros}
\index{macro}
Otra forma de acortar el código es mediante
\key{macros}.
Una macro significa que ciertas cadenas en
el código se cambiarán antes de la compilación.
En C++, las macros se definen utilizando la
palabra clave \texttt{\#define}.

Por ejemplo, podemos definir las siguientes macros:
\begin{lstlisting}
#define F first
#define S second
#define PB push_back
#define MP make_pair
\end{lstlisting}
Después de esto, el código
\begin{lstlisting}
v.push_back(make_pair(y1,x1));
v.push_back(make_pair(y2,x2));
int d = v[i].first+v[i].second;
\end{lstlisting}
puede abreviarse de la siguiente manera:
\begin{lstlisting}
v.PB(MP(y1,x1));
v.PB(MP(y2,x2));
int d = v[i].F+v[i].S;
\end{lstlisting}

Una macro también puede tener parámetros
que permiten abreviar ciclos y otras
estructuras de control.
Por ejemplo, podemos definir la siguiente macro:
\begin{lstlisting}
#define REP(i,a,b) for (int i = a; i <= b; i++)
\end{lstlisting}
Después de esto, el código
\begin{lstlisting}
for (int i = 1; i <= n; i++) {
    search(i);
}
\end{lstlisting}
puede abreviarse de la siguiente manera:
\begin{lstlisting}
REP(i,1,n) {
    search(i);
}
\end{lstlisting}

A veces, las macros causan errores que pueden resultar difíciles
detectar. Por ejemplo, considere la siguiente macro
que calcula el cuadrado de un número:
\begin{lstlisting}
#define SQ(a) a*a
\end{lstlisting}
Esta macro \emph{no} siempre funciona como se espera.
Por ejemplo, el código
\begin{lstlisting}
cout << SQ(3+3) << "\n";
\end{lstlisting}
corresponde al código
\begin{lstlisting}
cout << 3+3*3+3 << "\n"; // 15
\end{lstlisting}

Una mejor versión de la macro es la siguiente:
\begin{lstlisting}
#define SQ(a) (a)*(a)
\end{lstlisting}
Ahora el código
\begin{lstlisting}
cout << SQ(3+3) << "\n";
\end{lstlisting}
corresponde al código
\begin{lstlisting}
cout << (3+3)*(3+3) << "\n"; // 36
\end{lstlisting}


\section{Mathematics}

Mathematics plays an important role in competitive
programming, and it is not possible to become
a successful competitive programmer without
having good mathematical skills.
This section discusses some important
mathematical concepts and formulas that
are needed later in the book.

\subsubsection{Sum formulas}

Each sum of the form
\[\sum_{x=1}^n x^k = 1^k+2^k+3^k+\ldots+n^k,\]
where $k$ is a positive integer,
has a closed-form formula that is a
polynomial of degree $k+1$.
For example\footnote{\index{Faulhaber's formula}
There is even a general formula for such sums, called \key{Faulhaber's formula},
but it is too complex to be presented here.},
\[\sum_{x=1}^n x = 1+2+3+\ldots+n = \frac{n(n+1)}{2}\]
and
\[\sum_{x=1}^n x^2 = 1^2+2^2+3^2+\ldots+n^2 = \frac{n(n+1)(2n+1)}{6}.\]

An \key{arithmetic progression} is a \index{arithmetic progression}
sequence of numbers
where the difference between any two consecutive
numbers is constant.
For example,
\[3, 7, 11, 15\]
is an arithmetic progression with constant 4.
The sum of an arithmetic progression can be calculated
using the formula
\[\underbrace{a + \cdots + b}_{n \,\, \textrm{numbers}} = \frac{n(a+b)}{2}\]
where $a$ is the first number,
$b$ is the last number and
$n$ is the amount of numbers.
For example,
\[3+7+11+15=\frac{4 \cdot (3+15)}{2} = 36.\]
The formula is based on the fact
that the sum consists of $n$ numbers and
the value of each number is $(a+b)/2$ on average.

\index{geometric progression}
A \key{geometric progression} is a sequence
of numbers
where the ratio between any two consecutive
numbers is constant.
For example,
\[3,6,12,24\]
is a geometric progression with constant 2.
The sum of a geometric progression can be calculated
using the formula
\[a + ak + ak^2 + \cdots + b = \frac{bk-a}{k-1}\]
where $a$ is the first number,
$b$ is the last number and the
ratio between consecutive numbers is $k$.
For example,
\[3+6+12+24=\frac{24 \cdot 2 - 3}{2-1} = 45.\]

This formula can be derived as follows. Let
\[ S = a + ak + ak^2 + \cdots + b .\]
By multiplying both sides by $k$, we get
\[ kS = ak + ak^2 + ak^3 + \cdots + bk,\]
and solving the equation
\[ kS-S = bk-a\]
yields the formula.

A special case of a sum of a geometric progression is the formula
\[1+2+4+8+\ldots+2^{n-1}=2^n-1.\]

\index{harmonic sum}

A \key{harmonic sum} is a sum of the form
\[ \sum_{x=1}^n \frac{1}{x} = 1+\frac{1}{2}+\frac{1}{3}+\ldots+\frac{1}{n}.\]

An upper bound for a harmonic sum is $\log_2(n)+1$.
Namely, we can
modify each term $1/k$ so that $k$ becomes
the nearest power of two that does not exceed $k$.
For example, when $n=6$, we can estimate
the sum as follows:
\[ 1+\frac{1}{2}+\frac{1}{3}+\frac{1}{4}+\frac{1}{5}+\frac{1}{6} \le
1+\frac{1}{2}+\frac{1}{2}+\frac{1}{4}+\frac{1}{4}+\frac{1}{4}.\]
This upper bound consists of $\log_2(n)+1$ parts
($1$, $2 \cdot 1/2$, $4 \cdot 1/4$, etc.),
and the value of each part is at most 1.

\subsubsection{Set theory}

\index{set theory}
\index{set}
\index{intersection}
\index{union}
\index{difference}
\index{subset}
\index{universal set}
\index{complement}

A \key{set} is a collection of elements.
For example, the set
\[X=\{2,4,7\}\]
contains elements 2, 4 and 7.
The symbol $\emptyset$ denotes an empty set,
and $|S|$ denotes the size of a set $S$,
i.e., the number of elements in the set.
For example, in the above set, $|X|=3$.

If a set $S$ contains an element $x$,
we write $x \in S$,
and otherwise we write $x \notin S$.
For example, in the above set
\[4 \in X \hspace{10px}\textrm{and}\hspace{10px} 5 \notin X.\]

\begin{samepage}
New sets can be constructed using set operations:
\begin{itemize}
\item The \key{intersection} $A \cap B$ consists of elements
that are in both $A$ and $B$.
For example, if $A=\{1,2,5\}$ and $B=\{2,4\}$,
then $A \cap B = \{2\}$.
\item The \key{union} $A \cup B$ consists of elements
that are in $A$ or $B$ or both.
For example, if $A=\{3,7\}$ and $B=\{2,3,8\}$,
then $A \cup B = \{2,3,7,8\}$.
\item The \key{complement} $\bar A$ consists of elements
that are not in $A$.
The interpretation of a complement depends on
the \key{universal set}, which contains all possible elements.
For example, if $A=\{1,2,5,7\}$ and the universal set is
$\{1,2,\ldots,10\}$, then $\bar A = \{3,4,6,8,9,10\}$.
\item The \key{difference} $A \setminus B = A \cap \bar B$
consists of elements that are in $A$ but not in $B$.
Note that $B$ can contain elements that are not in $A$.
For example, if $A=\{2,3,7,8\}$ and $B=\{3,5,8\}$,
then $A \setminus B = \{2,7\}$.
\end{itemize}
\end{samepage}

If each element of $A$ also belongs to $S$,
we say that $A$ is a \key{subset} of $S$,
denoted by $A \subset S$.
A set $S$ always has $2^{|S|}$ subsets,
including the empty set.
For example, the subsets of the set $\{2,4,7\}$ are
\begin{center}
$\emptyset$,
$\{2\}$, $\{4\}$, $\{7\}$, $\{2,4\}$, $\{2,7\}$, $\{4,7\}$ and $\{2,4,7\}$.
\end{center}

Some often used sets are
$\mathbb{N}$ (natural numbers),
$\mathbb{Z}$ (integers),
$\mathbb{Q}$ (rational numbers) and
$\mathbb{R}$ (real numbers).
The set $\mathbb{N}$
can be defined in two ways, depending
on the situation:
either $\mathbb{N}=\{0,1,2,\ldots\}$
or $\mathbb{N}=\{1,2,3,...\}$.

We can also construct a set using a rule of the form
\[\{f(n) : n \in S\},\]
where $f(n)$ is some function.
This set contains all elements of the form $f(n)$,
where $n$ is an element in $S$.
For example, the set
\[X=\{2n : n \in \mathbb{Z}\}\]
contains all even integers.

\subsubsection{Logic}

\index{logic}
\index{negation}
\index{conjuction}
\index{disjunction}
\index{implication}
\index{equivalence}

The value of a logical expression is either
\key{true} (1) or \key{false} (0).
The most important logical operators are
$\lnot$ (\key{negation}),
$\land$ (\key{conjunction}),
$\lor$ (\key{disjunction}),
$\Rightarrow$ (\key{implication}) and
$\Leftrightarrow$ (\key{equivalence}).
The following table shows the meanings of these operators:

\begin{center}
\begin{tabular}{rr|rrrrrrr}
$A$ & $B$ & $\lnot A$ & $\lnot B$ & $A \land B$ & $A \lor B$ & $A \Rightarrow B$ & $A \Leftrightarrow B$ \\
\hline
0 & 0 & 1 & 1 & 0 & 0 & 1 & 1 \\
0 & 1 & 1 & 0 & 0 & 1 & 1 & 0 \\
1 & 0 & 0 & 1 & 0 & 1 & 0 & 0 \\
1 & 1 & 0 & 0 & 1 & 1 & 1 & 1 \\
\end{tabular}
\end{center}

The expression $\lnot A$ has the opposite value of $A$.
The expression $A \land B$ is true if both $A$ and $B$
are true,
and the expression $A \lor B$ is true if $A$ or $B$ or both
are true.
The expression $A \Rightarrow B$ is true
if whenever $A$ is true, also $B$ is true.
The expression $A \Leftrightarrow B$ is true
if $A$ and $B$ are both true or both false.

\index{predicate}

A \key{predicate} is an expression that is true or false
depending on its parameters.
Predicates are usually denoted by capital letters.
For example, we can define a predicate $P(x)$
that is true exactly when $x$ is a prime number.
Using this definition, $P(7)$ is true but $P(8)$ is false.

\index{quantifier}

A \key{quantifier} connects a logical expression
to the elements of a set.
The most important quantifiers are
$\forall$ (\key{for all}) and $\exists$ (\key{there is}).
For example,
\[\forall x (\exists y (y < x))\]
means that for each element $x$ in the set,
there is an element $y$ in the set
such that $y$ is smaller than $x$.
This is true in the set of integers,
but false in the set of natural numbers.

Using the notation described above,
we can express many kinds of logical propositions.
For example,
\[\forall x ((x>1 \land \lnot P(x)) \Rightarrow (\exists a (\exists b (a > 1 \land b > 1 \land x = ab))))\]
means that if a number $x$ is larger than 1
and not a prime number,
then there are numbers $a$ and $b$
that are larger than $1$ and whose product is $x$.
This proposition is true in the set of integers.

\subsubsection{Functions}

The function $\lfloor x \rfloor$ rounds the number $x$
down to an integer, and the function
$\lceil x \rceil$ rounds the number $x$
up to an integer. For example,
\[ \lfloor 3/2 \rfloor = 1 \hspace{10px} \textrm{and} \hspace{10px} \lceil 3/2 \rceil = 2.\]

The functions $\min(x_1,x_2,\ldots,x_n)$
and $\max(x_1,x_2,\ldots,x_n)$
give the smallest and largest of values
$x_1,x_2,\ldots,x_n$.
For example,
\[ \min(1,2,3)=1 \hspace{10px} \textrm{and} \hspace{10px} \max(1,2,3)=3.\]

\index{factorial}

The \key{factorial} $n!$ can be defined
\[\prod_{x=1}^n x = 1 \cdot 2 \cdot 3 \cdot \ldots \cdot n\]
or recursively
\[
\begin{array}{lcl}
0! & = & 1 \\
n! & = & n \cdot (n-1)! \\
\end{array}
\]

\index{Fibonacci number}

The \key{Fibonacci numbers}
%\footnote{Fibonacci (c. 1175--1250) was an Italian mathematician.}
arise in many situations.
They can be defined recursively as follows:
\[
\begin{array}{lcl}
f(0) & = & 0 \\
f(1) & = & 1 \\
f(n) & = & f(n-1)+f(n-2) \\
\end{array}
\]
The first Fibonacci numbers are
\[0, 1, 1, 2, 3, 5, 8, 13, 21, 34, 55, \ldots\]
There is also a closed-form formula
for calculating Fibonacci numbers, which is sometimes called
\index{Binet's formula} \key{Binet's formula}:
\[f(n)=\frac{(1 + \sqrt{5})^n - (1-\sqrt{5})^n}{2^n \sqrt{5}}.\]

\subsubsection{Logarithms}

\index{logarithm}

The \key{logarithm} of a number $x$
is denoted $\log_k(x)$, where $k$ is the base
of the logarithm.
According to the definition,
$\log_k(x)=a$ exactly when $k^a=x$.

A useful property of logarithms is
that $\log_k(x)$ equals the number of times
we have to divide $x$ by $k$ before we reach 
the number 1.
For example, $\log_2(32)=5$
because 5 divisions by 2 are needed:

\[32 \rightarrow 16 \rightarrow 8 \rightarrow 4 \rightarrow 2 \rightarrow 1 \]

Logarithms are often used in the analysis of
algorithms, because many efficient algorithms
halve something at each step.
Hence, we can estimate the efficiency of such algorithms
using logarithms.

The logarithm of a product is
\[\log_k(ab) = \log_k(a)+\log_k(b),\]
and consequently,
\[\log_k(x^n) = n \cdot \log_k(x).\]
In addition, the logarithm of a quotient is
\[\log_k\Big(\frac{a}{b}\Big) = \log_k(a)-\log_k(b).\]
Another useful formula is
\[\log_u(x) = \frac{\log_k(x)}{\log_k(u)},\]
and using this, it is possible to calculate
logarithms to any base if there is a way to
calculate logarithms to some fixed base.

\index{natural logarithm}

The \key{natural logarithm} $\ln(x)$ of a number $x$
is a logarithm whose base is $e \approx 2.71828$.
Another property of logarithms is that
the number of digits of an integer $x$ in base $b$ is
$\lfloor \log_b(x)+1 \rfloor$.
For example, the representation of
$123$ in base $2$ is 1111011 and
$\lfloor \log_2(123)+1 \rfloor = 7$.

\section{Contests and resources}

\subsubsection{IOI}

The International Olympiad in Informatics (IOI)
is an annual programming contest for
secondary school students.
Each country is allowed to send a team of
four students to the contest.
There are usually about 300 participants
from 80 countries.

The IOI consists of two five-hour long contests.
In both contests, the participants are asked to
solve three algorithm tasks of various difficulty.
The tasks are divided into subtasks,
each of which has an assigned score.
Even if the contestants are divided into teams,
they compete as individuals.

The IOI syllabus \cite{iois} regulates the topics
that may appear in IOI tasks.
Almost all the topics in the IOI syllabus
are covered by this book.

Participants for the IOI are selected through
national contests.
Before the IOI, many regional contests are organized,
such as the Baltic Olympiad in Informatics (BOI),
the Central European Olympiad in Informatics (CEOI)
and the Asia-Pacific Informatics Olympiad (APIO).

Some countries organize online practice contests
for future IOI participants,
such as the Croatian Open Competition in Informatics \cite{coci}
and the USA Computing Olympiad \cite{usaco}.
In addition, a large collection of problems from Polish contests
is available online \cite{main}.

\subsubsection{ICPC}

The International Collegiate Programming Contest (ICPC)
is an annual programming contest for university students.
Each team in the contest consists of three students,
and unlike in the IOI, the students work together;
there is only one computer available for each team.

The ICPC consists of several stages, and finally the
best teams are invited to the World Finals.
While there are tens of thousands of participants
in the contest, there are only a small number\footnote{The exact number of final
slots varies from year to year; in 2017, there were 133 final slots.} of final slots available,
so even advancing to the finals
is a great achievement in some regions.

In each ICPC contest, the teams have five hours of time to
solve about ten algorithm problems.
A solution to a problem is accepted only if it solves
all test cases efficiently.
During the contest, competitors may view the results of other teams,
but for the last hour the scoreboard is frozen and it
is not possible to see the results of the last submissions.

The topics that may appear at the ICPC are not so well
specified as those at the IOI.
In any case, it is clear that more knowledge is needed
at the ICPC, especially more mathematical skills.

\subsubsection{Online contests}

There are also many online contests that are open for everybody.
At the moment, the most active contest site is Codeforces,
which organizes contests about weekly.
In Codeforces, participants are divided into two divisions:
beginners compete in Div2 and more experienced programmers in Div1.
Other contest sites include AtCoder, CS Academy, HackerRank and Topcoder.

Some companies organize online contests with onsite finals.
Examples of such contests are Facebook Hacker Cup,
Google Code Jam and Yandex.Algorithm.
Of course, companies also use those contests for recruiting:
performing well in a contest is a good way to prove one's skills.

\subsubsection{Books}

There are already some books (besides this book) that
focus on competitive programming and algorithmic problem solving:

\begin{itemize}
\item S. S. Skiena and M. A. Revilla:
\emph{Programming Challenges: The Programming Contest Training Manual} \cite{ski03}
\item S. Halim and F. Halim:
\emph{Competitive Programming 3: The New Lower Bound of Programming Contests} \cite{hal13}
\item K. Diks et al.: \emph{Looking for a Challenge? The Ultimate Problem Set from
the University of Warsaw Programming Competitions} \cite{dik12}
\end{itemize}

The first two books are intended for beginners,
whereas the last book contains advanced material.

Of course, general algorithm books are also suitable for
competitive programmers.
Some popular books are:

\begin{itemize}
\item T. H. Cormen, C. E. Leiserson, R. L. Rivest and C. Stein:
\emph{Introduction to Algorithms} \cite{cor09}
\item J. Kleinberg and É. Tardos:
\emph{Algorithm Design} \cite{kle05}
\item S. S. Skiena:
\emph{The Algorithm Design Manual} \cite{ski08}
\end{itemize}
